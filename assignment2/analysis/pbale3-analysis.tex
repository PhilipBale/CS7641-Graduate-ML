 \documentclass[h]{article}
\usepackage[margin=0.5in]{geometry}
\usepackage{amsfonts} 
\usepackage{textcomp}
 
\usepackage{graphicx}
\usepackage{caption}
\usepackage{subcaption}
\usepackage{float} 
\usepackage{flafter}
\graphicspath{ {./plots/} }
\usepackage{adjustbox}


\newcommand{\cent}{\textcent \hspace{4pt}}
\title{CS 7641 Machine Learning \\ Assignment 2}
\date{Due Sunday March 11th, 2018 11:59pm}
\author{Philip Bale \\ pbale3}

\begin{document}

\maketitle

\section*{Part 1: Neural Network Optimization}
\subsection*{ Introduction}  
Part 1 of the assignment surrounds using randomized optimization to find the 
best possible weights for a specific neural network.  In assignment 1, backpropogation 
was used to find optimal parameters for a neural network.  This neural network 
took in various input features for US permanent visa applicants and then 
attempted to predict the outcome of an application.  After various tests, I found the optimal parameters of: 6-node input layer, one hidden layer with 100 nodes, 
one output node, and about 500 iterations.
\\ \\ I chose this problem because, as someone who has worked 
with a large number of first-generation visa holders and immigrants, I am 
extremely interested in building tools to help others to achieve the same.  At the end of the day, the goal is it to try to determine the application result 
before time, money, and other resources are spent.

\subsection*{1) Backpropogation (Assignment 1)}  
\subsubsection*{Overview}
The first weight-finding algorithm used was backpropogation.  Backpropogation works by essentially calculating 
the error at the end of a network, and then working backwards to minimize that error over various iterations. 
 An error (or loss) function is effectively minimized over time using this backpropogation technique.   As discussed in assignment 1, the permanent 
 visa is rather large and robust.  It is quickly learnable by various different 
 learners and in such backpropogation found significant success.
 
 \begin{figure}[H]
  \minipage{0.49\textwidth}
      \includegraphics[width=1\textwidth,keepaspectratio]{backprop_nn.jpg} 
      \caption*{Permanent Visa Applicant NN Learning Curve} 
   \endminipage\hfill
   \minipage{0.49\textwidth}
      \includegraphics[width=1\textwidth,keepaspectratio]{backprop_nn_time.jpg} 
      \caption*{Permanent Visa Applicant NN Learning Curve} 
   \endminipage\hfill
\end{figure}

Right around 50 iterations, the network begins to converge at around an 80\% 
successs rate.  Seeing as the training and test score track either rather 
closely, it is apparent that the dataset is rather robust and consistent.  One 
thing to note is that the training time scales linearly with the number of 
iterations--which makes sense since the same amount of calculations with similar 
complexity are performed on each iteration of backpropogation.  

\subsection*{2) Randomized Hill Climbiing}  
\subsubsection*{Overview}
The second weight-finding algorithm used was randomized hill climbing.  
Randomized hill climbing works by taking a random starting point and then 
incrementally attempting to improve on that point.  In the context of a neural 
network trying to find weights,  randomized hill climbing selects random weights 
and then moves in a direction so as to try to find a better result for that 
weight--akin to trying to move up an optimization 'success hill'.
\\ \\ 
One thing to 
note is that we are using randomized hill climbing, not random restart hill 
climbing.  In such, the algorithm is prone to getting caught in local optimizations, or local maximums. 


 \begin{figure}[H]
  \minipage{0.49\textwidth}
      \includegraphics[width=1\textwidth,keepaspectratio]{randomized_hill_climbing_nn.jpg} 
      \caption*{Permanent Visa Applicant NN Learning Curve} 
   \endminipage\hfill
   \minipage{0.49\textwidth}
      \includegraphics[width=1\textwidth,keepaspectratio]{randomized_hill_climbing_nn_time.jpg} 
      \caption*{Permanent Visa Applicant NN Learning Curve} 
   \endminipage\hfill
\end{figure}
High accuracy results are achieved right around 125 iterations+.  
While this is subject to randomness, by looking at the results it is shown to 
become rather consistant.  By looking at the score results, one can see various 
instances of the randomized hill climbing getting caught in local optimas and 
being unable to escape.  Such is the case around 220, 320, and 350 iterations.
\\ \\
Similar to backpropogation, training time scales linearly with the number of 
 iterations run. Training time tends to be a bit faster using randomized hill 
 climbing because of a reducation in calculations necessary.  Whereas 
 backpropogation needed to do calcluations to minimize error moving backwards 
 through the network, randomized hill climbing simply needs to move in one 
 direction and determine if the new weights are better.
 
\subsection*{3) Simulated Annealing}  
\subsubsection*{Overview}
The third weight-finding algorithm used was simulated annealilng.

\subsection*{4) Genetic Algorithms}  
\subsubsection*{Overview}
The fourth weight-finding algorithm used was genetic algorithm..

\subsection*{ Conclusion}  
asdf

\section*{Part 2: Optimization Problem Domains}
\subsection*{ Introduction}  
asdf

\subsection*{1) TSP}  
\subsubsection*{Overview}
Asdf

\subsection*{2) TSP}  
\subsubsection*{Overview}
Asdf

\subsection*{3) TSP}  
\subsubsection*{Overview}
Asdf

\subsection*{ Conclusion}  
asdf

\begin{itemize}
  \item Job features: Industry code, job class, wage rate, wage type
  \item Geographic features: Country of citizen and employer location
\end{itemize}
The classes observed for this dataset are simply 'approved' and 'denied'.  The 
dataset contains 374365 total samples.
\subsubsection*{Why is the dataset interesting?}
This dataset is interesting due to its potential to aid in the visa application 
process from a cost and time savings potential.  It could also enable confidence in those 
interested in applying for a US permanent visa but doubting their chances of 
acceptance.  At the end of the day, the goal is it to try to determine the application result 
before time, money, and other resources are spent.  As someone who has worked 
with a large number of first-generation visa holders and immigrants, I am 
extremely interested in building tools to help others to achieve the same.
\\ \\
From a machine learning perspective, the dataset is incredibly interesting due 
to its wide variety of features and the variety of values those features can take. 
 An immense number of job types, wage rates, and citizenships alone create an 
 extremely diverse dataset.  Additionally, the number of samples available 
 provide a comprehensive picture of historical data, lending towards greater 
 cofidence in training and testing rates.

\subsection*{2) Home Sale Price Predictions}  
\subsubsection*{Overview}
The second classification problem revolves around classifying a home's price 
bracket based upon the various characteristics of the home.  Among the features 
used in the classification are :
\begin{itemize}
  \item Subjective measurements: Exterior condition, house style, overall quality rating, and overall condition
  \item Objective measurements: Type of dwelling, building type, lot size, neighborhood, year built, and year sold
\end{itemize}
After an initial review of the dataset, the classes were defined as pricing 
brackets divided into 100k groups.  I.e: 0-100k, 100k-200k, 200k-300k, etc.  The 
dataset contains 1451 samples.  An additional dataset containing another 1400 
testing samples exists but was not used as it contains unclassified sale 
prices.  It will, however, prove useful for unsupervised learning.
\subsubsection*{Why is the dataset interesting?}
This dataset is interesting for two primary reasons: real-world applicability 
and participating in a Kaggle challenge.  First, modeling home prices is both a 
difficult and lucrative task.  If one can succesfully model home sale prices on 
large sets of data, he/she can make large amounts of money investing in real 
estate when he/she detects outliers in listed price vs. what it is expected to sell for.  
This applies to flipping, investing, and remodeling.  Second, the dataset is 
part of an ongoing Kaggle competition that does not have a winning solution yet. 
 By taking part of the competition, the dataset presents the opportunity to work 
 towards a winning solution and advance ones algorithms over time. 
 \\ \\
 Houses can have a very large amount of features--with a large amount of variety 
 in the individual features.  Similarly, housing is prone to personal taste and 
 frequent need for upgrades/modernization.  In such, I believe price estimation is an 
 excellent problem, full of depth and complexity, that is suitable for a machine 
 learning approach.
 
  \section*{General Data Processing}
  The datasets I used were both relatively clean to begin with.  One small 
  problem, however, was that a lot of my features on both datasets were 
  text-based.  To transform the features into numeric values suitable for the 
  machine learning algorithms, I used a label encoder built into ScikitLearn.
  \\ \\
  I also did a small amount of preprocessing of the data to make it more 
  suitable for classification.  I dropped all unnecessary columns to help 
  speed up with data processing in general--which proved immensely helpful when 
  dealing with the more computationall intensive algorithms.  In the 
  case of home prices, I precalculated the brackets based on the 'sale price' data label.  
  In the case of visa applications, I pregrouped the case outcome so that 
  results such as 'certified' and 'certified withdrawn' are both concerned as 
  'approved' conditions whereas 'denied', 'invalidated', and 'rejected' all 
  resolve to 'denied'.
  
 \section*{#1: Decision Trees}
 A decision tree classifier was the first algorithm applied to the datasets.  
 Various values of max\_depth were tested as a means of pruning unnecessary 
 leaves.  Similarly, a grid search was used to test whether a 'gini' or 
 'entropy' criterion was more effective.
%
%\subsection*{US Permanent Visa Data}
%\begin{figure}[H]
%\minipage{0.69\textwidth}
%\begin{tabular}{ | c | c  | c | c | c | c | c |} 
%\hline
%\textbf{Depth} & \textbf{Criterion} & \textbf{Tree Size} & \textbf{Train \%} & \textbf{Train Time} & \textbf{Test \%} & \textbf{Test Time}   \\
%\hline
%1 & gini & 3 & 0.7657 & 0.1212 & 0.7680 & 0.0010 \\ \hline
%3 & gini & 15 & 0.6775 & 0.1417 & 0.6749 & 0.0009 \\ \hline
%6 & entropy & 105 & 0.7121 & 0.2605 & 0.6924 & 0.0011 \\ \hline
%10 & gini & 889 & 0.7584 & 0.3192 & 0.7072 & 0.0011 \\ \hline
%15 & gini & 3013 & 0.8628 & 0.4199 & 0.7582 & 0.0014 \\ \hline
%20 & gini & 4751 & 0.9299 & 0.4517 & 0.7917 & 0.0010 \\ \hline
%25 & gini & 5349 & 0.9545 & 0.5284 & 0.8032 & 0.0010 \\ \hline
%35 & entropy & 5349 & 0.9574 & 0.5116 & 0.8081 & 0.0010 \\ \hline
%\end{tabular}
%\caption*{Results at multiple depths for best criterion via grid search}
%\endminipage\hfill
%\minipage{0.29\textwidth}
%\begin{flushright}
%\begin{tabular}{ | c | c | c  | } 
%\hline
% & Accepted & Denied  \\
%\hline
%Accepted & 96 & 135 \\ \hline
%Denied & 217 & 1376 \\ \hline
%\end{tabular}
%\caption*{Test Data Confusion matrix}
%\end{flushright}
%\endminipage\hfill
%
%\end{figure}
%
%\begin{figure}[H]
%  \minipage{0.32\textwidth}
%      \includegraphics[width=1\textwidth,keepaspectratio]{1_curve_dtree3.png} 
%      \caption*{Learning Curve for max\_depth = 3} 
%   \endminipage\hfill
%   \minipage{0.32\textwidth}
%      \includegraphics[width=1\textwidth,keepaspectratio]{1_curve_dtree10.png} 
%      \caption*{Learning Curve for max\_depth = 10} 
%   \endminipage\hfill
%   \minipage{0.32\textwidth}
%      \includegraphics[width=1\textwidth,keepaspectratio]{1_curve_dtree20.png} 
%      \caption*{Learning Curve for max\_depth = 20} 
%   \endminipage\hfill
%\end{figure}
%
%\subsection*{Housing Prices Data}
%\begin{figure}[H]
%\minipage{0.69\textwidth}
%\begin{tabular}{ | c | c  | c | c | c | c | c |} 
%\hline
%\textbf{Depth} & \textbf{Criterion} & \textbf{TreeSize} & \textbf{Train \%} & \textbf{Train Time} & \textbf{Test \%} & \textbf{Test Time}   \\
%\hline
%1 & gini & 3 & 0.0912 & 0.0355 & 0.0890 & 0.0006 \\ \hline
%3 & entropy & 15 & 0.5816 & 0.0401 & 0.5890 & 0.0003 \\ \hline
%6 & entropy & 91 & 0.6820 & 0.0442 & 0.6438 & 0.0003 \\ \hline
%10 & gini & 331 & 0.8575 & 0.0566 & 0.6918 & 0.0006 \\ \hline
%15 & gini & 593 & 0.9854 & 0.0695 & 0.7603 & 0.0003 \\ \hline
%20 & gini & 643 & 1.0000 & 0.0559 & 0.7603 & 0.0007 \\ \hline
%25 & gini & 639 & 1.0000 & 0.0550 & 0.7603 & 0.0006 \\ \hline
%35 & gini & 639 & 1.0000 & 0.0570 & 0.7603 & 0.0006 \\ \hline
%
%\end{tabular}
%\caption*{Results at multiple depths for best criterion via grid search}
%\endminipage\hfill
%\minipage{0.29\textwidth}
%\begin{flushright}
%\begin{tabular}{ | c | c | c  | c | c | c | } 
%\hline
% & 0-1 & 1-2 & 2-3 & 3-4 & 4-5  \\
%\hline
%0-1 & 2 & 6 & 0 & 0 & 0 \\ \hline 
%1-2 & 4 & 87 & 7 & 0 & 0 \\ \hline 
%2-3 & 0 & 8 & 15 & 1 & 2 \\ \hline 
%3-4 & 0 & 0 & 6 & 3 & 0 \\ \hline 
%4-5 & 0 & 0 & 2 & 0 & 3 \\ \hline 
%\end{tabular}
%\caption*{  \ \ \  Test Data Confusion matrix  \\  \ \ \ \ \ \ (classes in 100ks)}
%\end{flushright}
%\endminipage\hfill
%
%\end{figure}
%
%\begin{figure}[H]
%  \minipage{0.32\textwidth}
%      \includegraphics[width=1\textwidth,keepaspectratio]{2_curve_dtree3.png} 
%      \caption*{Learning Curve for max\_depth = 3} 
%   \endminipage\hfill
%   \minipage{0.32\textwidth}
%      \includegraphics[width=1\textwidth,keepaspectratio]{2_curve_dtree10.png} 
%      \caption*{Learning Curve for max\_depth = 10} 
%   \endminipage\hfill
%   \minipage{0.32\textwidth}
%      \includegraphics[width=1\textwidth,keepaspectratio]{2_curve_dtree20.png} 
%      \caption*{Learning Curve for max\_depth = 20} 
%   \endminipage\hfill
%\end{figure}
%
%\subsection*{Analysis for Decision Tree}
%Overall the classifier worked quite well for both datasets, but was extremely prone to 
%overfitting.  Examining the results of the decision tree classifier on the two 
%datasets provides numerous observations and basis for analysis, which are provided below.
%
%\subsubsection*{Effects of dataset size & cross validation:}
%Above, learning curves are provided for both datasets.  It is immediately 
%apparent that the size of the dataset greatly affects the performance of the 
%algorithm--though diminishes over time. This makes sense for a few different 
%reasons.  The more data we have to train on, the more likely it is that 
%we see the full spectrum of possible variability.  Similarly, the broader the 
%set of examples, the less biased our algorithm will be.  This is because if we 
%only train on a few data samples, then our algorithm can only make decisions 
%based on the features learned from the small, simple sample size--thus generating a bias (and therefore 
%underfitting).
%\\ \\ 
%The learning curves for both datasets clearly level out as 
%dataset size increases, demonstrating less variance with a more informed model.  
%Similarly, cross-validation was used to normalize the training 
%samples and smooth the learnign curve.  Without cross-validationg, the model was 
%prone to unrepresentative, biased dataset samples during training and, in-turn, testing.
%\subsubsection*{Tuning Parameters:}
%The two main paremeters tuned were maximum tree depth and the split criterion.  
%The split criterion measures the quality of a tree split.  Both 'gini' and 
%'entropy' were tested and evaluated using a gridsearch.  For the large majority 
%of trials, especially with larger and more complex trees, 'gini' was the more 
%effective splitting criterion--though the results for the two were nearly 
%identical.  If anything, Gini was more performant due to it's mathematical 
%simplicity over the entropy formula.
%\\ \\
%Tree depth provided to be a singificant influencer over the performance of our 
%model--especially for the housing dataset.  By allowing for a greater tree 
%depth, we allow for a more complex tree with more possible decisions.  A problem 
%emerges when the tree becomes too deep, however.  Instead of becoming more 
%robust, the tree begins to overfit with very specific branches for very specific 
%data items.  By analyzing the test \% vs. depth tradeoff, we can prune 
%unnecessary tree depths for the optimal tree size.  This allowed for both a 
%fast, accurate tree with a reasonable footprint.
%
%\subsubsection*{Performance:}
%Peformance varied for the two datasets across a few different domains.  In terms of runtime, housing prices took 
%longer than the visa data to train (about 10x)--though both were extremely fast on a powerful laptop.
%  This was mainly due to a larger dataset size.  As 
%the depth of the tree increased, the training time also took longer for both datasets due to the
%increased tree size and complexity(as can be seen from the table above).
%\\ \\ 
%In terms of accuracy, the visa data started out somewhat high but was able to gain about 4\% through 
%optimizing the depth of the tree.  This is due to the fact that features like 
%'salary' and 'job type' proved to heavily influence the approval process.  The 
%housing data, however, saw more greater improvements as depth increased.    As a 
%more diverse and variable dataset, the model was able to gain a lot of accuracy 
%as the tree grew because it could accommodate more specific feature branches.  While the test data
%started out at \textless 10\%, it was able to grow to approximately 76\%.   
%
%\begin{figure}[H]
%    \includegraphics[width=1.0\textwidth,keepaspectratio]{1_dtree.jpg} 
%    \caption*{SampleDecision Tree: Permanent Visa with max\_depth of 7} 
%\end{figure}
%
%\section*{#2) AdaBoost - Boosted Trees}
%A boosted decision tree classifer, Adaboost, was the second algorithm applied.  It used the same 
%base decision tree as the first model algorithm.  Based on the results of the 
%first model, we decided to stick with 'gini' as the criterion for faster 
%training.
%
%\subsection*{US Permanent Visa Data}
%\begin{figure}[H]
%\minipage{0.69\textwidth}
%\begin{tabular}{ | c | c  | c | c | c | c | c |} 
%\hline
%\textbf{Depth} & \textbf{Learning Rate} & \textbf{\# Estimators} & \textbf{Train \%} & \textbf{Train Time} & \textbf{Test \%} & \textbf{Test Time}   \\
%1 & 0.1 & 150 & 0.8705 & 24.1484 & 0.8662 & 0.0327 \\ \hline
%3 & 0.1 & 100 & 0.8770 & 39.7266 & 0.8679 & 0.0216 \\ \hline
%5 & 0.1 & 5 & 0.8761 & 53.8227 & 0.8706 & 0.0021 \\ \hline
%10 & 0.1 & 150 & 0.9775 & 82.2192 & 0.8745 & 0.0504 \\ \hline
%15 & 1 & 150 & 0.9775 & 94.5996 & 0.8701 & 0.0420 \\ \hline
%20 & 1 & 150 & 0.9775 & 107.0742 & 0.8618 & 0.0452 \\ \hline
%\hline
%\end{tabular}
%\caption*{Results at multiple depths for best learning rate/# estimators via grid search}
%\endminipage\hfill
%\end{figure}
%
%\begin{figure}[H]
%  \minipage{0.32\textwidth}
%      \includegraphics[width=1\textwidth,keepaspectratio]{1_curve_boost3.png} 
%      \caption*{Learning Curve for max\_depth = 3} 
%   \endminipage\hfill
%   \minipage{0.32\textwidth}
%      \includegraphics[width=1\textwidth,keepaspectratio]{1_curve_boost10.png} 
%      \caption*{Learning Curve for max\_depth = 10} 
%   \endminipage\hfill
%   \minipage{0.32\textwidth}
%      \includegraphics[width=1\textwidth,keepaspectratio]{1_curve_boost20.png} 
%      \caption*{Learning Curve for max\_depth = 20} 
%   \endminipage\hfill
%\end{figure}
%
%\subsection*{Housing Prices Data}
%\begin{figure}[H]
%\minipage{0.69\textwidth}
%\begin{tabular}{ | c | c  | c | c | c | c | c |} 
%\hline
%\textbf{Depth} & \textbf{Learning Rate} & \textbf{\# Estimators} & \textbf{Train \%} & \textbf{Train Time} & \textbf{Test \%} & \textbf{Test Time}   \\
%\hline
%1 & 0.1 & 15 & 0.6881 & 6.9413 & 0.6918 & 0.0018 \\ \hline
%3 & 0.1 & 5 & 0.7724 & 8.2909 & 0.7945 & 0.0013 \\ \hline
%5 & 0.1 & 3 & 0.8291 & 10.8559 & 0.7740 & 0.0011 \\ \hline
%10 & 1 & 50 & 1.0000 & 13.6713 & 0.7945 & 0.0057 \\ \hline
%15 & 1 & 100 & 1.0000 & 8.7526 & 0.8014 & 0.0145 \\ \hline
%20 & 0.1 & 15 & 1.0000 & 0.6862 & 0.6781 & 0.0009 \\ \hline
%
%\end{tabular}
%\caption*{Results at multiple depths for best learning rate/# estimators via grid search}
%\endminipage\hfill
%
%\end{figure}
%
%\begin{figure}[H]
%  \minipage{0.32\textwidth}
%      \includegraphics[width=1\textwidth,keepaspectratio]{2_curve_boost3.png} 
%      \caption*{Learning Curve for max\_depth = 3} 
%   \endminipage\hfill
%   \minipage{0.32\textwidth}
%      \includegraphics[width=1\textwidth,keepaspectratio]{2_curve_boost10.png} 
%      \caption*{Learning Curve for max\_depth = 10} 
%   \endminipage\hfill
%   \minipage{0.32\textwidth}
%      \includegraphics[width=1\textwidth,keepaspectratio]{2_curve_boost20.png} 
%      \caption*{Learning Curve for max\_depth = 20} 
%   \endminipage\hfill
%\end{figure}
%
%\subsection*{Analysis for AdaBoost Boosted Tree}
%The AdaBoost boosted tree relates significantly to the original decision tree 
%classifier.  It even uses the decision tree as its base classifier.  The main 
%difference, however, exists in the boosting.  As discussed in class, a boosted 
%tree is essentially an ensemble of weaker trees.  This ensemble works to 
%classify with greater accuracy than the original, simplified tree. 
%\\ \\
%Evaluating the performance in the chart and graphs above, it is quickly noticed that the AdaBoost tree works
%significantly better for both datasets.  For visa applications, it is approximately 7\% better in the optimal 
%configuration.  For housing prices, it i approximately 4\% better.  Training 
%times take signifcantly longer, which makes sense as the algorithm trains a much 
%larger, ensemble classifier.  For example, the larger permanent visa dataset took more than a minute to
% train for depths greater than 5--compared to less than a second for a normal 
% tree classifier.
% \\ \\
% In terms of parameters, learning rate, depth, and # of estimators were all 
% tuned using a gridsearch to find an optimal model.  As # of estimators 
% increased, optimal learning rate tended to increase too.  For a deeper tree, 
% fewer estimators were needed because the optimal learning rate was achieved 
% sooner.  The learning rate, which effects the rate of contribution for each 
% classifier, was optimized between 0.1 and 1, where the learning rate trended 
% higher for deeper trees.
% \\ \\
% While both types of trees peformed well on the datasets, AdaBoost performed 
% exceptionally well.  I believe this is due to the rich set of features and logically 
% classifiable outcomes.  The AdaBoosted tree still overclassified at 
% increased depth.  By pruning depth to approximately 10 to 15, we were able to achieve 
% an optimal tree.
% 
% \section*{#3) Neural Networks}
%A neural network (using ScikitLearn's multi-layer perceptron classifier) was the third algorithm applied.  
%A combination of different solvers, learning rates, and scaling was used to 
%observe the functionality of the networks in regards to our dataset.
%
%
%\begin{figure}[H]
%  \minipage{0.49\textwidth}
%      \includegraphics[width=1\textwidth,keepaspectratio]{1_nn.png} 
%      \caption*{Permanent Visa Applicant NN Learning Curve} 
%   \endminipage\hfill
%   \minipage{0.49\textwidth}
%      \includegraphics[width=1\textwidth,keepaspectratio]{2_nn.png} 
%      \caption*{Housing Price NN Learning Curve} 
%   \endminipage\hfill
%\end{figure}
%
%\begin{figure}[H]
%  \minipage{1\textwidth}
%      \includegraphics[width=1\textwidth,keepaspectratio]{1_nn_together.png} 
%      \caption*{Permanent Visa Applicant NN Configurations Loss Curve} 
%   \endminipage\hfill
%   \minipage{1\textwidth}
%      \includegraphics[width=1\textwidth,keepaspectratio]{2_nn_together.png} 
%      \caption*{Housing Price NN Configurations Loss Curve} 
%   \endminipage\hfill
%\end{figure}
%
%\begin{figure}[H]
%\minipage{0.69\textwidth}
%\begin{tabular}{ | c | c  | c | c | c | c | c |} 
%\hline
%\textbf{NN Config} & \textbf{Test Score} & \textbf{Loss} & \textbf{Train Time}   \\ 
%\hline
%constant learning-rate & 0.8642 & 0.3842 & 0.6121 \\ \hline
%sgd constant  learning-rate with momentum & 0.8663 & 0.3823 & 0.7120 \\ \hline
%sgd constant  learning-rate with Nesterov's momentum & 0.8668 & 0.3779 & 1.4836 \\ \hline
%sgd inv-scaling learning-rate & 0.8646 & 0.3905 & 0.2435 \\ \hline
%sgd inv-scaling with momentum & 0.8646 & 0.3842 & 0.3750 \\ \hline
%sgd inv-scaling with Nesterov's momentum & 0.8646 & 0.3848 & 0.2740 \\ \hline
%adam constant  learning-rate & 0.8677 & 0.3650 & 2.1166 \\ \hline
%\end{tabular}
%\caption*{Permanent Visa results for various NN configs}
%\endminipage\hfill
%\end{figure}
%
%\begin{figure}[H]
%\minipage{0.69\textwidth}
%\begin{tabular}{ | c | c  | c | c | c | c | c |} 
%\hline
%\textbf{NN Config} & \textbf{Test Score} & \textbf{Loss} & \textbf{Train Time}   \\ 
%\hline
%sgd constant learning-rate & 0.7664 & 0.6301 & 0.4272  \\ \hline
%sgd constant learning-rate with momentum & 0.7498 & 0.6058 & 0.0889 \\ \hline
%sgd constant  learning-ratewith Nesterov's momentum & 0.7595 & 0.5669 & 0.1286 \\ \hline
%sgd inv-scaling learning-rate & 0.6278 & 1.0219 & 0.3932  \\ \hline
%sgd inv-scaling with momentum & 0.7009 & 0.7936 & 1.9339  \\ \hline
%sgd inv-scaling with Nesterov's momentum & 0.6278 & 1.2652 & 0.0205  \\ \hline
%adam constant  learning-rate  & 0.7795 & 0.5237 & 0.2093  \\ \hline
%\end{tabular}
%\caption*{Housing Price results for various NN configs}
%\endminipage\hfill
%\end{figure}
%
%\subsection*{Analysis for Neural Networks}
%The neural network models, while effective, performed slightly worse 
%than the boosted decision tree.  After preliminary exploration, it became 
%apparent that the wide variety of neural network configurations performed 
%differently depending on the given  dataset.
%\\ \\
%In terms of training and testing, cross-validation was used to ensure a robust 
%sampling.  The housing price dataset started to converge slowly in regards to 
%the learning curve, whereas the permanent visa dataste convered very rapidly.  
%The permanent visa dataset, by converging rapidly, showed it's low variability.  
%The network model was able to very quickly learn the features--in less than 40--though its feature set is
% smaller than those of the housing prices.  The housing price dataset, 
% conversely, required many more iterations to learn.  
% \\ \\
% Various neural network configurations were used on the test data.  Configurations were created by 
% varying the weight optimizer, learning-rate, momentum, and scaling.  To aid 
% with speed and peformance, a pre-process Scaler was applied to the datasets.
% Additionally, a gridsearch approach was applied with various alpha, learning 
% rate, solver, and hidden layer conigurations.  While complete results achieved by the various networks are listed
%  above, below we will look at the top performing network for each dataset to try to 
%  understand why it was most effective.
%  \\ \\
%  For the permanent visa dataset, practically every algorithm performed the same.  At first, this perplexed me, 
%  as I expected that at least different learning rates and solvers would create 
%  different results.  It turns out, however, for a dataset that is more easily 
%  learnable and converges on fewer iterations, this makes sense.  It takes less 
%  effort and hyperparemeterization to find functional solutions.  To verify that 
%  my results were accurate, I also ran a gridsearched neural network with a 2000 
%  iteration limit and various values of solver, learning\_rate, and hidden 
%  layers.
%  \\ \\
%  For the housing price dataset, the 'adam' weight optimizer significantly outperformed the others.  
%  This was quite different than the Permanent Visa dataset, which performed 
%  rather homogenously.  The 'adam' optimizer, or solver, is a 'stochastic 
%  gradient-based optimizer.' (From SciKit learn manual)  It is noted to work 
%  especially well for large datasets.  The learning rate was slightly smaller 
%  for this configuration, starting with an initial value of 0.01.   Since the 
%  housing price network took longer to converge and performed quite differently 
%  for the various configuration, it is reasonable to believe that with a larger 
%  dataset, a more complex and accurate network could be trained.  While this 
%  would increase training time, the trade off of increased accuracy and lower 
%  variance would likely be possible.
%
% \section*{#4) Support Vector Machines}
%Support Vector Machines were the fourth algorithm applied. 
%
%\subsection*{US Permanent Visa Data}
%
%\begin{figure}[H]
%\minipage{0.69\textwidth}
%\begin{tabular}{ | c | c  | c | c | c | c | c |} 
%\hline
%\textbf{Kernel} & \textbf{Gamma} & \textbf{Train \%} & \textbf{Train Time} & \textbf{Test \%} & \textbf{Test Time}   \\ \hline
%rbf & 0.01 & 0.8642 & 20.4459 & 0.8712 & 0.1536 \\ \hline
%rbf & 0.05 & 0.8646 & 29.0756 & 0.8712 & 0.1817 \\ \hline
%rbf & 1.0 & 0.8752 & 29.7723 & 0.8745 & 0.1805 \\ \hline
%rbf & 2.0 & 0.8798 & 28.6184 & 0.8728 & 0.1914 \\ \hline
%poly & 0.01 & 0.8642 & 9.3467 & 0.8712 & 0.0884 \\ \hline
%poly & 0.05 & 0.8645 & 20.6110 & 0.8712 & 0.0857 \\ \hline
%poly & 1.0 & 0.8301 & 24.1879 & 0.8262 & 0.1198 \\ \hline
%poly & 2.0 & 0.8349 & 19.5531 & 0.8355 & 0.0391 \\ \hline
%
%\hline
%\end{tabular}
%\caption*{Permanent Visa Data for various kernel/gamma configurations}
%\endminipage\hfill
%\end{figure}
%
%\begin{figure}[H]
%  \minipage{0.32\textwidth}
%      \includegraphics[width=1\textwidth,keepaspectratio]{1_svm_rbf_05.png} 
%      \caption*{RBF SVM w/ 0.05 gamma} 
%   \endminipage\hfill
%   \minipage{0.32\textwidth}
%      \includegraphics[width=1\textwidth,keepaspectratio]{1_svm_rbf_1.png} 
%      \caption*{RBF SVM w/ 1.0 gamma} 
%   \endminipage\hfill
%   \minipage{0.32\textwidth}
%      \includegraphics[width=1\textwidth,keepaspectratio]{1_svm_rbf_2.png} 
%      \caption*{RBF SVM w/ 2.0 gamma} 
%   \endminipage\hfill
%\end{figure}
%
%\begin{figure}[H]
%  \minipage{0.32\textwidth}
%      \includegraphics[width=1\textwidth,keepaspectratio]{1_svm_poly_05.png} 
%      \caption*{Poly SVM w/ 0.05 gamma} 
%   \endminipage\hfill
%   \minipage{0.32\textwidth}
%      \includegraphics[width=1\textwidth,keepaspectratio]{1_svm_poly_1.png} 
%      \caption*{Poly SVM w/ 1.0 gamma} 
%   \endminipage\hfill
%   \minipage{0.32\textwidth}
%      \includegraphics[width=1\textwidth,keepaspectratio]{1_svm_poly_2.png} 
%      \caption*{Poly SVM w/ 2.0 gamma} 
%   \endminipage\hfill
%\end{figure}
%
%\subsection*{Housing Price Data}
%
%\begin{figure}[H]
%\minipage{0.69\textwidth}
%\begin{tabular}{ | c | c  | c | c | c | c | c |} 
%\hline
%\textbf{Kernel} & \textbf{Gamma} & \textbf{Train \%} & \textbf{Train Time} & \textbf{Test \%} & \textbf{Test Time}   \\ \hline
%rbf & 0.01 & 0.7487 & 0.2301 & 0.7603 & 0.0042 \\ \hline
%rbf & 0.05 & 0.7954 & 0.2031 & 0.8219 & 0.0086 \\ \hline
%rbf & 1.0 & 0.9226 & 0.5629 & 0.7740 & 0.0042 \\ \hline
%rbf & 2.0 & 0.9502 & 0.5996 & 0.7397 & 0.0051 \\ \hline
%poly & 0.01 & 0.6276 & 0.1581 & 0.6438 & 0.0024 \\ \hline
%poly & 0.05 & 0.7333 & 0.1733 & 0.7534 & 0.0028 \\ \hline
%poly & 1.0 & 0.9333 & 0.7254 & 0.7192 & 0.0019 \\ \hline
%poly & 2.0 & 0.9280 & 1.0244 & 0.6781 & 0.0019 \\ \hline
%
%
%\hline
%\end{tabular}
%\caption*{Housing Price Data for various kernel/gamma configurations}
%\endminipage\hfill
%\end{figure}
%
%\begin{figure}[H]
%  \minipage{0.32\textwidth}
%      \includegraphics[width=1\textwidth,keepaspectratio]{2_svm_rbf_01.png} 
%      \caption*{RBF SVM w/ 0.01 gamma} 
%   \endminipage\hfill
%   \minipage{0.32\textwidth}
%      \includegraphics[width=1\textwidth,keepaspectratio]{2_svm_rbf_05.png} 
%      \caption*{RBF SVM w/ 0.05 gamma} 
%   \endminipage\hfill
%   \minipage{0.32\textwidth}
%      \includegraphics[width=1\textwidth,keepaspectratio]{2_svm_rbf_1.png} 
%      \caption*{RBF SVM w/ 1.0 gamma} 
%   \endminipage\hfill
%\end{figure}
%\begin{figure}[H]
%  \minipage{0.32\textwidth}
%      \includegraphics[width=1\textwidth,keepaspectratio]{2_svm_poly_01.png} 
%      \caption*{Poly SVM w/ 0.01 gamma} 
%   \endminipage\hfill
%   \minipage{0.32\textwidth}
%      \includegraphics[width=1\textwidth,keepaspectratio]{2_svm_poly_05.png} 
%      \caption*{Poly SVM w/ 0.05 gamma} 
%   \endminipage\hfill
%   \minipage{0.32\textwidth}
%      \includegraphics[width=1\textwidth,keepaspectratio]{2_svm_poly_1.png} 
%      \caption*{Poly SVM w/ 1.0 gamma} 
%   \endminipage\hfill
%\end{figure}
%
%
%\subsection*{Analysis for Support Vector Machines}
%The Support Vector Machines peformed well compared to the other algorithms, 
%though, on-average, took significantly longer to train.  Overall, the optimal configurations yieled a testing rate of 87.5\% for the 
%permanent visa data and 77.4\% for the housing price data.     As there were a lot of 
%parameters to tune, a gridsearch was used yet again along with cross-validation 
%to ensure a robust and effective model.  The primary parameters were kernel and 
%gamma.
%\\ \\
%To begin, two different  kernels were used for each dataset: 'rbf' and 'poly'.  
%For both datasets, the rbf kernel proved more effective.  The gaussian rbf 
%kernel essentially calculates squared distance between feature vectors and is 
%scaled by gamma.  While the poly kernel tends to overfit the training data, rbf 
%does a much better job at securing a robust fit.  To tune Gamma, values on a 
%spectrum from 0.01 to 2.0 were used (and results detailed above).
%\\ \\
%Training time increased signficantly, especially for larger datasets.  Since a 
%SVMs require a large amount of iterations and  recalculation to converge to an 
%optimal solution, this made sense.  It was especially apparent for the larger 
%dataset of permanent visa data.  The training time took approximately 30 seconds 
%for the various configurations where as the smaller housing price dataset took 
%less than a second on average.
%\\ \\
%From the learning curves, we see that a highly succesful training rate leads to 
%greater overfitting.  This makes sense, as the SVM becomes too rigid to the 
%input data, and is not robust.  Similarly, when the kernel has too low of a 
%gamma and does not accomodate feature relations properly, the model underfits 
%and performs poorly on both training and test data.
%
% \section*{#5) K-nearest Neighbors}
%K-nearest Neighbors was the fifth algorithm applied. 
%
%\subsection*{Permanent Visa Data}
%
%\begin{figure}[H]
%\minipage{0.6\textwidth}
%\begin{tabular}{ | c | c  | c | c | c | c | c |} 
%\hline
%
%\textbf{Weight} & \textbf{K} & \textbf{Train \%} & \textbf{Train Time} & \textbf{Test \%} & \textbf{Test Time}   \\ \hline
%uniform & 1 & 0.9708 & 0.3422 & 0.8158 & 0.0080 \\ \hline
%uniform & 2 & 0.8978 & 0.3619 & 0.7708 & 0.0094 \\ \hline
%uniform & 3 & 0.8989 & 0.3603 & 0.8394 & 0.0106 \\ \hline
%uniform & 4 & 0.8854 & 0.3836 & 0.8202 & 0.0103 \\ \hline
%uniform & 5 & 0.8892 & 0.3996 & 0.8503 & 0.0107 \\ \hline
%uniform & 10 & 0.8742 & 0.4736 & 0.8525 & 0.0126 \\ \hline
%uniform & 15 & 0.8739 & 0.5735 & 0.8640 & 0.0134 \\ \hline
%uniform & 20 & 0.8738 & 0.6083 & 0.8673 & 0.0168 \\ \hline
%uniform & 30 & 0.8696 & 0.8306 & 0.8662 & 0.0219 \\ \hline
%uniform & 50 & 0.8686 & 1.1392 & 0.8618 & 0.0315 \\ \hline
%distance & 1 & 0.9708 & 0.2706 & 0.8158 & 0.0071 \\ \hline
%distance & 2 & 0.9675 & 0.4111 & 0.8103 & 0.0113 \\ \hline
%distance & 3 & 0.9743 & 0.4050 & 0.8279 & 0.0096 \\ \hline
%distance & 4 & 0.9741 & 0.3766 & 0.8311 & 0.0092 \\ \hline
%distance & 5 & 0.9756 & 0.4084 & 0.8377 & 0.0157 \\ \hline
%distance & 10 & 0.9762 & 0.4899 & 0.8470 & 0.0142 \\ \hline
%distance & 15 & 0.9767 & 0.6011 & 0.8509 & 0.0169 \\ \hline
%distance & 20 & 0.9782 & 0.6799 & 0.8591 & 0.0290 \\ \hline
%distance & 30 & 0.9782 & 0.9327 & 0.8624 & 0.0255 \\ \hline
%distance & 50 & 0.9782 & 1.2837 & 0.8635 & 0.0397 \\ \hline
%
%\hline
%\end{tabular}
%\caption*{Permanent Visa Data}
%\endminipage\hfill
%  \minipage{0.45\textwidth}
%      \includegraphics[width=1\textwidth,keepaspectratio]{1_knn_compare.png} 
%      \caption*{KNN Comparision} 
%   \endminipage\hfill
%\end{figure}
% 
%
%\begin{figure}[H]
%  \minipage{0.32\textwidth}
%      \includegraphics[width=1\textwidth,keepaspectratio]{1_knn_1_1.png} 
%      \caption*{Uniform KNN - 1} 
%   \endminipage\hfill
%   \minipage{0.32\textwidth}
%      \includegraphics[width=1\textwidth,keepaspectratio]{1_knn_3_1.png} 
%      \caption*{Uniform KNN - 3} 
%   \endminipage\hfill
%   \minipage{0.32\textwidth}
%      \includegraphics[width=1\textwidth,keepaspectratio]{1_knn_20_1.png} 
%      \caption*{Uniform KNN - 20} 
%   \endminipage\hfill
%\end{figure}
%\begin{figure}[H]
%  \minipage{0.32\textwidth}
%      \includegraphics[width=1\textwidth,keepaspectratio]{1_knn_1_2.png} 
%      \caption*{Distance KNN - 1} 
%   \endminipage\hfill
%   \minipage{0.32\textwidth}
%      \includegraphics[width=1\textwidth,keepaspectratio]{1_knn_3_2.png} 
%      \caption*{Distance KNN - 3} 
%   \endminipage\hfill
%   \minipage{0.32\textwidth}
%      \includegraphics[width=1\textwidth,keepaspectratio]{1_knn_20_2.png} 
%      \caption*{Distance KNN - 20} 
%   \endminipage\hfill
%\end{figure}
%
%\subsection*{Housing Price Data}
%
%\begin{figure}[H]
%\minipage{0.6\textwidth}
%\begin{tabular}{ | c | c  | c | c | c | c | c |} 
%\hline
%
%\textbf{Weight} & \textbf{K} & \textbf{Train \%} & \textbf{Train Time} & \textbf{Test \%} & \textbf{Test Time}   \\ \hline
%uniform & 1 & 1.0000 & 0.0343 & 0.6507 & 0.0015 \\ \hline
%uniform & 2 & 0.8222 & 0.0304 & 0.6644 & 0.0011 \\ \hline
%uniform & 3 & 0.8130 & 0.0283 & 0.7055 & 0.0011 \\ \hline
%uniform & 4 & 0.7824 & 0.0379 & 0.6986 & 0.0012 \\ \hline
%uniform & 5 & 0.7602 & 0.0351 & 0.7260 & 0.0012 \\ \hline
%uniform & 10 & 0.7341 & 0.0324 & 0.7055 & 0.0019 \\ \hline
%uniform & 15 & 0.7218 & 0.0369 & 0.7192 & 0.0016 \\ \hline
%uniform & 20 & 0.7134 & 0.0466 & 0.7329 & 0.0013 \\ \hline
%uniform & 30 & 0.7004 & 0.0488 & 0.7055 & 0.0022 \\ \hline
%uniform & 50 & 0.6812 & 0.0635 & 0.6986 & 0.0020 \\ \hline
%distance & 1 & 1.0000 & 0.0244 & 0.6507 & 0.0009 \\ \hline
%distance & 2 & 1.0000 & 0.0261 & 0.6575 & 0.0010 \\ \hline
%distance & 3 & 1.0000 & 0.0293 & 0.7055 & 0.0010 \\ \hline
%distance & 4 & 1.0000 & 0.0282 & 0.6986 & 0.0010 \\ \hline
%distance & 5 & 1.0000 & 0.0346 & 0.7397 & 0.0011 \\ \hline
%distance & 10 & 1.0000 & 0.0356 & 0.7123 & 0.0012 \\ \hline
%distance & 15 & 1.0000 & 0.0385 & 0.7192 & 0.0017 \\ \hline
%distance & 20 & 1.0000 & 0.0468 & 0.7534 & 0.0021 \\ \hline
%distance & 30 & 1.0000 & 0.0501 & 0.7260 & 0.0016 \\ \hline
%distance & 50 & 1.0000 & 0.0742 & 0.7055 & 0.0021 \\ \hline
%
%\hline
%\end{tabular}
%\caption*{Housing Price Data}
%\endminipage\hfill
%  \minipage{0.45\textwidth}
%      \includegraphics[width=1\textwidth,keepaspectratio]{2_knn_compare.png} 
%      \caption*{KNN Comparision} 
%   \endminipage\hfill
%\end{figure}
% 
%
%\begin{figure}[H]
%  \minipage{0.32\textwidth}
%      \includegraphics[width=1\textwidth,keepaspectratio]{2_knn_1_1.png} 
%      \caption*{Uniform KNN - 1} 
%   \endminipage\hfill
%   \minipage{0.32\textwidth}
%      \includegraphics[width=1\textwidth,keepaspectratio]{2_knn_3_1.png} 
%      \caption*{Uniform KNN - 3} 
%   \endminipage\hfill
%   \minipage{0.32\textwidth}
%      \includegraphics[width=1\textwidth,keepaspectratio]{2_knn_20_1.png} 
%      \caption*{Uniform KNN - 20} 
%   \endminipage\hfill
%\end{figure}
%\begin{figure}[H]
%  \minipage{0.32\textwidth}
%      \includegraphics[width=1\textwidth,keepaspectratio]{2_knn_1_2.png} 
%      \caption*{Distance KNN - 1} 
%   \endminipage\hfill
%   \minipage{0.32\textwidth}
%      \includegraphics[width=1\textwidth,keepaspectratio]{2_knn_3_2.png} 
%      \caption*{Distance KNN - 3} 
%   \endminipage\hfill
%   \minipage{0.32\textwidth}
%      \includegraphics[width=1\textwidth,keepaspectratio]{2_knn_20_2.png} 
%      \caption*{Distance KNN - 20} 
%   \endminipage\hfill
%\end{figure}
%
%\subsection*{Analysis for k-Nearest Neighbors}
%k-Nearest neighbors was by far the simplest model employed, but it also 
%performed relatively poorly on the smaller housing dataset.  The primary reason for this was 
%overfitting.  Using a distance weight generally improved data quality for both datasets. 
% In terms of training and testing time, it was extremely fast for both datasets, as there 
% was very little calculation necessary.
%\\ \\
%Two different weight functions were used to measure the contribution of a k-Nearest 
%neighbor: uniform and distance.  While uniform treated each neighbor equally, 
%distance had closer neighbords take priority.  This worked well, as it 
%essentially treated more spatially similar results in testing as likely to be more 
%accurate.  While unfiform did not perform horrendously, it tended to underfit 
%the data--especially for the housing dataset, due to a high variance in the 
%features.
%\\ \\
%A k-value of approximately 20 performed best for both datasets, as can be seen 
%using the kNN comparison graphs. The housing price data, with a smaller dataset, 
%tended to perform worse using kNN due to the smaller number of items to train 
%on and the greater likelihood of values from a different class being taken into 
%account as a neighbor--whereas the permanent visa data, with a large dataset, 
%was more likely to have same-class values contribute as a neighbor. 
%
%\subsection*{Conclusion}
%In total, 5 different machine learning algorithms were applied to two datasets: 
%US Permanent Visa applications and Housing Prices for a small geographic region 
%in the US.  Each algorithm maintained its pros/cons that varied heavily on the 
%type of model and its parameters.  By consistantly using gridsearch and 
%cross-validation, various relatively robust models were trained for each 
%dataset.  All-in-all, Support Vector Machines performed the best on both 
%datasets in terms of testing accuracy, though did require a large amount of training relative to the other 
%models for that dataset.  The SVM, by using it's kernel to accurately relate features across training and test sets, 
%was able to set accurate decision boundaries for the data--performing well for 
%both datasets, but especially well for the larger visa dataset.
%\\ \\
%\begin{figure}[H]
%\minipage{0.6\textwidth}
%\begin{tabular}{ | c | c  | c | c | c | c | c |} 
%\hline
%
%\textbf{Dataset} & \textbf{Model} & \textbf{Specifications} & \textbf{Test \%} & \textbf{Train Time}   \\ \hline
%Visa & Decision Tree & criterion=gini, depth=25 &  80.32\% & 0.5284 \\ \hline
%Visa & Boosted DT & depth=10, learning rate=0.1 & 87.45\% & 82.2192 \\ \hline
%Visa & Neural Net & solver=adam, learning rate init = 0.01 & 86.77\% & 2.1166 \\ \hline
%Visa & SVM & kernel=rbf, gamma=1.0 & 87.45\% & 29.7723 \\ \hline
%Visa & kNN & weight=uniform, k=20 & 86.73\% & 0.6083  \\ \hline
%
%Housing & Decision Tree & criterion=gini, depth=15 & 76.03\% & 0.0695  \\ \hline
%Housing & Boosted DT & depth=15, learning rate=1 & 80.14\% & 8.75  \\ \hline
%Housing & Neural Net & solver=adam, learning rate init = 0.01 & 77.95\% & 0.2093 \\ \hline
%Housing & SVM & kernel=rbf, gamma=0.05 & 82.19\% & 0.2031  \\ \hline
%Housing & kNN & weight=distance, k=20 & 75.34\% & 0.0468  \\ \hline
%\hline
%\end{tabular}
%\caption*{Optimal configuration for each algorithms, by dataset}
%   \endminipage\hfill
%\end{figure}



\end{document}